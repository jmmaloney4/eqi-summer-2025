\documentclass[11pt]{amsart}

% -----------------------------------------------------------
% PACKAGES
\usepackage{amssymb, amsmath}
\usepackage{hyperref}
\usepackage{geometry}
\geometry{margin=1in}

% -----------------------------------------------------------
% THEOREM STYLES
\newtheorem{theorem}{Theorem}[section]
\newtheorem{lemma}[theorem]{Lemma}
\newtheorem{corollary}[theorem]{Corollary}
\newtheorem{proposition}[theorem]{Proposition}

\theoremstyle{definition}
\newtheorem{definition}[theorem]{Definition}
\newtheorem{remark}[theorem]{Remark}
\newtheorem{example}[theorem]{Example}

\theoremstyle{remark}
\newtheorem*{note}{Note}

% -----------------------------------------------------------
% CUSTOM COMMANDS
\newcommand{\R}{\mathbb{R}}
\newcommand{\C}{\mathbb{C}}

% -----------------------------------------------------------
% DOCUMENT START
\begin{document}

\title{Notes on \textit{Elements of Quantitative Investing} by Paleologo}
\author{Jack Maloney}
% \address{Your Department, University, City, Country}
\email{jmmaloney4@gmail.com}
% \date{\today}

% \begin{abstract}
% This is a short abstract describing the main results of the paper.
% \end{abstract}

\maketitle

\section{Probability Spaces and Measurable Functions}

\begin{definition}
A \emph{probability space} is a triple $(\Omega, \mathcal{F}, P)$ where $\Omega$ is a set, $\mathcal{F}$ is a $\sigma$-algebra of subsets of $\Omega$, and $P: \mathcal{F} \to [0,1]$ is a probability measure with $P(\Omega) = 1$.
\end{definition}

\begin{definition}
A \emph{random variable} is a measurable function $X: (\Omega, \mathcal{F}) \to (\mathbb{R}, \mathcal{B}(\mathbb{R}))$, where $\mathcal{B}(\mathbb{R})$ denotes the Borel $\sigma$-algebra on $\mathbb{R}$.
\end{definition}

\begin{remark}
This abstract formulation highlights the seamless extension of measure-theoretic integration to random variables, setting the stage for advanced probabilistic and functional-analytic constructions.
\end{remark}

\section{Integration and Expectation}

\begin{definition}
For an integrable random variable $X$, its \emph{expectation} (or \emph{mean}) is defined as the Lebesgue integral
\[
\mathbb{E}[X] := \int_\Omega X(\omega) \, dP(\omega),
\]
provided the integral exists in the extended real line.
\end{definition}

\begin{definition}
Let $\mathcal{G} \subseteq \mathcal{F}$ be a sub-$\sigma$-algebra. The \emph{conditional expectation} of $X$ given $\mathcal{G}$, denoted $\mathbb{E}[X|\mathcal{G}]$, is the unique (up to almost sure equality) $\mathcal{G}$-measurable function satisfying
\[
\int_G \mathbb{E}[X|\mathcal{G}] \, dP = \int_G X \, dP, \quad \forall G \in \mathcal{G}.
\]
\end{definition}

\begin{remark}
In $L^2(\Omega, \mathcal{F}, P)$, $\mathbb{E}[X|\mathcal{G}]$ coincides with the orthogonal projection of $X$ onto the closed subspace of $\mathcal{G}$-measurable square-integrable random variables.
\end{remark}

\section{\texorpdfstring{$L^p$}{Lp} Spaces and Equivalence Classes}

\begin{definition}
For $1 \leq p < \infty$, the space $L^p(\Omega, \mathcal{F}, P)$ consists of equivalence classes of measurable functions $X$ with finite $p$-th moment:
\[
\|X\|_p := \left( \int_\Omega |X|^p \, dP \right)^{1/p} < \infty.
\]
\end{definition}

\begin{definition}
Two random variables $X, Y$ are said to be \emph{equivalent} if $P(X \neq Y)=0$. Each element of $L^p$ is an equivalence class $[X]$ of almost surely equal functions.
\end{definition}

\begin{lemma}[Existence of Borel-Measurable Representatives]
Let $X \in L^p(\Omega, \mathcal{F}, P)$. There exists a Borel-measurable representative $Y \in [X]$ that coincides with $X$ almost surely. However, no canonical representative exists in general.
\end{lemma}

\begin{note}
In applications, representatives with additional regularity, such as continuity or cadlag paths, are often chosen to facilitate analysis and computations.
\end{note}

\begin{proposition}[Radon-Nikodym Derivative as Probability Density]
Let $X$ be a random variable and let $P_X = X_*P$ denote its distribution. If $P_X \ll \lambda$ (the Lebesgue measure), then there exists a density function $f_X \in L^1(\mathbb{R})$ such that
\[
P_X(B) = \int_B f_X(x) \, dx, \quad \forall B \in \mathcal{B}(\mathbb{R}).
\]
This $f_X = \frac{dP_X}{d\lambda}$ is the \emph{probability density function} (pdf) of $X$.
\end{proposition}

\section{Inner Product Structure and Covariance}

\begin{definition}
The space $L^2(\Omega, \mathcal{F}, P)$ is a Hilbert space with inner product
\[
\langle X, Y \rangle := \int_\Omega X Y \, dP.
\]
\end{definition}

\begin{definition}
For centered random variables (i.e., $\mathbb{E}[X]=0$), the \emph{covariance} is
\[
\operatorname{Cov}(X, Y) = \langle X, Y \rangle.
\]
\end{definition}

\section{Moments and Generating Functions}

\begin{definition}
The \emph{$n$-th moment} of a random variable $X$ is $\mathbb{E}[X^n]$, provided the expectation exists.
\end{definition}

\begin{theorem}[Hölder's Inequality]
Let $1 \leq p, q \leq \infty$ with $\frac{1}{p} + \frac{1}{q} = 1$. Then
\[
|\mathbb{E}[XY]| \leq \|X\|_p \|Y\|_q.
\]
\end{theorem}

\begin{definition}
The \emph{moment generating function} (MGF) of $X$, if it exists in a neighborhood of zero, is
\[
M_X(t) := \mathbb{E}[e^{tX}].
\]
\end{definition}

\section{Modes of Convergence}

\begin{definition}
Let $\{X_n\}$ be a sequence of random variables. We distinguish several modes of convergence:
\begin{enumerate}[label=(\roman*)]
\item \emph{Almost sure convergence}: $X_n(\omega) \to X(\omega)$ for $P$-almost all $\omega$.
\item \emph{Convergence in probability}: $\forall \varepsilon>0$, $P(|X_n - X| > \varepsilon) \to 0$.
\item \emph{$L^p$ convergence}: $\|X_n - X\|_p \to 0$.
\item \emph{Convergence in distribution}: $P_{X_n} \to P_X$ weakly.
\end{enumerate}
\end{definition}

\begin{theorem}[Prokhorov's Theorem]
A family of probability measures on a Polish space is tight if and only if it is relatively compact in the topology of weak convergence.
\end{theorem}

\begin{theorem}[Skorokhod's Representation Theorem]
Let $P_{X_n} \to P_X$ weakly. Then there exist random variables $\{\tilde{X}_n\}$ and $\tilde{X}$ on a common probability space such that $\tilde{X}_n \xrightarrow{a.s.} \tilde{X}$ and $\tilde{X}_n$ has the same distribution as $X_n$ for each $n$.
\end{theorem}

\section{Functional Analytic Perspectives}

\begin{theorem}[Riesz Representation Theorem]
Let $1 < p < \infty$. Every continuous linear functional on $L^p(\Omega, \mathcal{F}, P)$ is of the form
\[
\varphi(X) = \int_\Omega X Y \, dP
\]
for some $Y \in L^q(\Omega, \mathcal{F}, P)$ with $\frac{1}{p} + \frac{1}{q}=1$.
\end{theorem}

\begin{remark}
This perspective frames the expectation as a continuous linear functional and highlights the duality between $L^p$ and $L^q$ spaces.
\end{remark}


% -----------------------------------------------------------
% REFERENCES
\begin{thebibliography}{9}

\bibitem{ref1}
Author Name, \emph{Title of Paper}, Journal Name, vol. XX, pp. 1--10, 20XX.

\end{thebibliography}

\end{document}
