\documentclass[11pt]{amsart}

% -----------------------------------------------------------
% PACKAGES
\usepackage{amssymb, amsmath}
\usepackage{hyperref}
\usepackage{geometry}
\geometry{margin=1in}

% -----------------------------------------------------------
% THEOREM STYLES
\newtheorem{theorem}{Theorem}[section]
\newtheorem{lemma}[theorem]{Lemma}
\newtheorem{corollary}[theorem]{Corollary}
\newtheorem{proposition}[theorem]{Proposition}

\theoremstyle{definition}
\newtheorem{definition}[theorem]{Definition}

\theoremstyle{remark}
\newtheorem{remark}[theorem]{Remark}

% -----------------------------------------------------------
% CUSTOM COMMANDS
\newcommand{\R}{\mathbb{R}}
\newcommand{\C}{\mathbb{C}}

% -----------------------------------------------------------
% DOCUMENT START
\begin{document}

\title{Notes on \textit{Elements of Quantitative Investing} by Paleologo}
\author{Jack Maloney}
% \address{Your Department, University, City, Country}
\email{jmmaloney4@gmail.com}
% \date{\today}

% \begin{abstract}
% This is a short abstract describing the main results of the paper.
% \end{abstract}

\maketitle

\section{,.}

Here is the introduction...

\section{Main Results}

\begin{theorem}
Let $f: \R \to \R$ be continuous. Then ...
\end{theorem}

\begin{proof}
Sketch of the proof...
\end{proof}

\section{Conclusion}

Some concluding remarks.

% -----------------------------------------------------------
% REFERENCES
\begin{thebibliography}{9}

\bibitem{ref1}
Author Name, \emph{Title of Paper}, Journal Name, vol. XX, pp. 1--10, 20XX.

\end{thebibliography}

\end{document}
